\section{Inledning}
\label{sec:inledning}
För att anpassa sig till dygnets cykliska beteende använder sig många organismer av en så kallad cirkadisk rytm, vilket är en biologisk klocka med en period på 24 timmar. I en förenklad modell över hur denna klocka fungerar kan rytmen sägas bero på två specifika, reglerande proteiner, ett som undertrycker (repressor, i rapporten nämnd protein R) och ett som aktiverar (activator, protein A) relevanta processer. Undersökningar visar att oscillatorns beteende till största del beror på två faktorer: koncentrationen av protein R och molekylärdynamiken i processen då protein A bildar ett inaktivt komplex med R \cite{ref:rapport}.

Syftet med detta miniprojekt var att utnyttja \emph{Gillespies algoritm} på det studerade systemet för att avgöra hur den stokastiska metodens resultat skiljer sig från den tidigare använda \cite{ref:bestarapportenyao} deterministiska metoden. Resultaten för två olika värden på nedbrytningshastigheten för protein R skulle jämföras och allmänna skillnader mellan de både metoderna skulle studeras.

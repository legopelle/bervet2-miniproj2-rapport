\section{Numeriska metoder}
\label{sec:metoder}

\begin{figure}
	\centering
	\resizebox{\textwidth}{!}{
		\begin{tikzpicture}[node distance=4cm, auto]
		\node [block](init) {Initialize model \\*
			$t \leftarrow t_0$ \\*
			$x \leftarrow x_0$};
		\node [block, right of=init] (timestep) {Roll for inter reaction time $\tau$};
		\node [block, right of=timestep] (reaction) {Roll for reaction $r$};
		\node [block, right of=reaction] (execute) {Carry out reaction \\*
			$x \leftarrow x + n_r$ };
		\node [block, right of=execute] (increase) {Update time \\*
			$t \leftarrow t + \tau$};
		\node [decision, right of=increase] (condition) {Check \\*
			$t < t_\mathrm{final}$};
		\node [block, right of=condition] (done) {Done};
		\coordinate [below of=condition] (null);
		
		
		\draw [-latex'] (init) -- (timestep);
		\draw [-latex'] (timestep) -- (reaction);
		\draw [-latex'] (reaction) -- (execute);
		\draw [-latex'] (execute) -- (increase);
		\draw [-latex'] (increase) -- (condition);
		\draw [-latex'] (condition) -- node [above] {False} (done);
		\draw [-latex'] (condition) -- node [right] {True} (null) -| (timestep);
		
		\end{tikzpicture}
	}
	\caption{Schematisk beskrivning av \emph{Gillespies algoritm}. I varje steg slumpas ett tidssteg $\tau \sim \exp(\beta)$ och en reaktion $r$ viktat efter deras propensiteter. Slutligen uppdateras tiden $t$ och tillståndet $x$.}
	\label{fig:gill}
\end{figure}

Till skillnad från \textsc{miniprojekt 1} där vi med ordinära
differentialekvationer arbetade deterministiskt, använde vi nu en
stokastisk metod. Detta krävde dock en omskrivning av den matematiska
modellen (se avsnitt ??).

Den ursprungliga kontinuerliga och deterministiska metoden baserades på
lösning av differentialekvationer som beskrev koncentrationen av de intressanta
partiklarna. Detta antar att antalet partiklar är stort så att koncentrationen
är approximativt kontinuerlig. I en cell är dock det antagandet svårt att
motivera, ty antalet av varje partikel kan finnas i endast ett fåtal exemplar.

Vi använde instället den diskreta och stokastiska \emph{Gillespies algoritm}. En beskrivning av algoritmen ses i figur~\ref{fig:gill}. Nu arbetar vi
direkt med partiklars antal istället för koncentrationer, och istället för
reaktionshastigheter har vi sannolikheter (propensiteter) för varje reaktion.

Först initierar vi vårt tillstånd med initialvärden för tiden $t$ (\lstinline|t|) och tillståndet $x$ (\lstinline|x|) enligt avsnitt~??. Sedan beräknar vi propensiteterna $w_r$ enligt \lstinline|w = prop_vilar(x, b)| där \lstinline|b| är en parametervektor enligt avsnitt~?? och \lstinline|w| är en $18 \times 1$ kolumnvektor. Vi kan då beräkna summan $\beta$ som \lstinline|beta = sum(w)|.

Sedan slumpar vi fram $\tau$ med hjälp av att invertera den kumulativa sannolikhetsfördelningen. Detta görs med ett slumptal $u_1 \sim U[0,1]$ enligt
\begin{equation}
\tau = \frac{-\ln u_1}{\beta} \, .
\end{equation}
Detta implementerar vi i matlab som
\begin{lstlisting}
u1 = rand;
tau = -log(u1)/beta;
\end{lstlisting}




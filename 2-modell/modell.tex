\section{Matematisk modell}
\label{sec:modell}

Den cirkadiska rytmen utgörs av 18 reaktioner som beskrivs i ekvation~20 (sidan 21) i \cite{hellander}. Varje reaktion $r$ har en tillhörande propensitet $w_r$ som vi kommer använda som vikt i slumpat urval. Varje reaktion kan beskrivas algebraiskt som en differens $n_r$ som beskriver hur varje partikel förbrukas eller produceras.

Propensiteterna $w_r$ är funktioner av tillståndet $x$. Givet $w_r$ för varje reaktion kan vi tala om den \emph{totala propensiteten} $\beta$ som är summan av varje enskild reaktions propensitet. Då kommer väntetiden mellan två reaktioner $\tau$ vara exponentialfördelad med parameter $\beta$.